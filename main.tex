\documentclass[10pt, dvipdfmx]{jarticle}
%\usetheme{Copenhagen}
%\usepackage{color}
%\usefonttheme{professionalfonts}
\usepackage{amsmath, bm}
\usepackage{mathrsfs}
\renewcommand{\kanjifamilydefault}{\gtdefault}
\newcommand{\dif}[2]{\frac{\text d #1}{\text d #2}}
\newcommand{\pdif}[2]{\frac{\partial #1}{\partial #2}}
\newcommand{\la}{\left <}
\newcommand{\ra}{\right >}
\newcommand{\kb}{k_\text{B}}
\newcommand{\dt}{\Delta t}

% pdfの栞が文字化けするのを防ぐ
\usepackage{atbegshi}
\ifnum 42146=\euc"A4A2 \AtBeginShipoutFirst{\special{pdf:tounicode EUC-UCS2}}\else
\AtBeginShipoutFirst{\special{pdf:tounicode 90ms-RKSJ-UCS2}}\fi

%\setbeamertemplate{navigation symbols}{}

% \onlide< N - M > と書きます. 
% N番目からM番目までのスライドで表示されるようになります.
% それ以外のところでは表示されません. 


\begin{document}
よく書かれている一般化Langevin方程式などで用いられる.
\begin{equation}\label{eq:F_Langevin_form_0}
    F_\text{w}(t) = \int_0^t \lambda(t-s) u_\text{s}(s) \text{d} s + \delta F_\text{w}(t)
\end{equation}
において積分の下限が$-\infty$ではなく$0$としばしば見なされる理由を考える.
これは,一般には基準時刻$0$によって$F_\text{w}(t)$の値が変化してしまうため正しくないと思われ,
物理的には下限を$-\infty$にした
\begin{equation}\label{eq:F_Langevin_form_infty}
    F_\text{w}(t) = \int_{-\infty}^t \lambda(t-s) u_\text{s}(s) \text{d} s + \delta F_\text{w}(t)
\end{equation}
とするべきである.
実は,式\eqref{eq:F_Langevin_form_0}の両辺と$F_\text{w}(0)$との相関をとったとみられる形の
\begin{equation}\label{eq:F_cor_Langevin_form_0}
    \la F_\text{w}(0) F_\text{w}(t) \ra = \int_0^t \lambda(t-s) \la F_\text{w}(0) u_\text{s}(s) \ra \text{d} s
\end{equation}
が成り立つ.
これを式\eqref{eq:F_Langevin_form_infty}を用いて証明する.

\end{document}
